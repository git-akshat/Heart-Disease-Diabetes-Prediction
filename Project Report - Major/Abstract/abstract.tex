\begin{abstracts}        
\normalsize{
\begin{enumerate}
\item \textbf{Objectives of the project} Paragraph on motivation to do the current project
	\begin{itemize}
		\item The proposed system predicts heart diseases as well as the chances of diabetes
		\item There are no proper methods to handle semi structured and unstructured data. The proposed system is expected to work well with both structured and unstructured data.
		\item The secondary target of the project is to build up a web application which permits clients to anticipate diabetes and coronary illness by using prediction engine.
\end{itemize}
\item \textbf{Description} With the advance of big data analytics equipment, more devotion has been paid to disease expectation from the perception of big data inquiry, various explores have been conducted by choosing the features mechanically from a large number of data to improve the truth of menace classification rather than the formerly selected physiognomies. However, those prevailing work mostly measured structured data. Number of inquires about has been led to choosing the characteristics of a disease prediction from an enormous volume of an data. A large portion of the current work is based on structured data. For the unstructured data one can utilize a convolutional neural network. Convolutional neural network are comprised of a neurons, every neuron gets a few sources of info and performs activities and the entire network expresses a single differentiable score methods.   \\
The system analyses the structured and unstructured data in healthcare field to assess the risk of disease. First, it uses Decision tree map algorithm to generate the pattern and causes of disease. Second, by using Map Reduce algorithm for partitioning the data such that a query will be analyzed only in a specific partition, which will increase the operational efficiency but reduce query retrieval time. Map reducing algorithm is used for partitioning the medical data based on the output of Decision Tree map algorithm. Compared to several typical prediction algorithms, the prediction accuracy of our proposed algorithm increases.
\item \textbf{Validation of Test Results}  
\begin{itemize}
		\item \textbf{K-Fold Cross-Validation:} In K-fold cross validation, the data set is separated into K equivalent size of parts, in which K less one groups are utilized to train the classifiers and remaining part is utilized for checking outperformance in each progression. The procedure of validation is repeated K times. The classifier performance is computed dependent on K results..
		\item \textbf{Manual Validation:} The test results predicted by the model can be cross verified with the observed patient medical reports
\end{itemize}
\item Process used to solve the problem
\item Objectives achieved and Results summary
\end{enumerate}

}
\end{abstracts}

% ----------------------------------------------------------------------

